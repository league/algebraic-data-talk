%%%_* Document preamble
\documentclass[14pt,t,usepdftitle=false,
xcolornames=x11names,svgnames,dvipsnames]{beamer}
\usepackage{euler}
\usepackage{fontspec}
\usepackage{listings}
\usepackage{alltt}
\usepackage{ulem}
\usepackage{multirow}
%%%_* Simplest theme ever; white background, no widgets
\usetheme{default}
\setbeamertemplate{navigation symbols}{}
\setbeameroption{show notes}
\setbeamertemplate{frametitle}[default][center]
\setbeamersize{text margin left=10mm}
%%%_** Fancy fonts
\setbeamerfont{frametitle}{series=\bfseries}
\setmainfont[Mapping=tex-text]{Warnock Pro}
\setsansfont[Mapping=tex-text,Numbers={OldStyle}]{Cronos Pro}
\setmonofont[Mapping=tex-text]{Inconsolata}
\newcommand{\wackyFont}[1]{
  {\LARGE\fontspec[Mapping=tex-text]{Immi Five O Five Std} #1}}
\newcommand{\subtitleFont}[1]{{\footnotesize #1}}
\newcommand{\slideheading}[1]{
  \begin{center}
    \usebeamerfont{frametitle}
    \usebeamercolor[fg]{frametitle}#1
  \end{center}\vskip-5mm}
\usefonttheme{professionalfonts}
%%%_** Color definitions
\colorlet{comment}{Olive}
\colorlet{string}{SaddleBrown}
\colorlet{keyword}{Navy}
\colorlet{type}{Green}
\colorlet{emph}{Maroon}
\colorlet{input}{Indigo}
\colorlet{error}{DarkRed}
\colorlet{result}{LightSlateGrey}
\colorlet{background}{LightGoldenrodYellow}
\colorlet{hole}{LimeGreen}
\newcommand{\cd}[1]{\textcolor{emph}{\texttt{#1}}}
%%%_** Listings settings
% "define" Scala
\lstset{
  language=Haskell,
  columns=flexible,
  showstringspaces=false,
  basicstyle={\small\ttfamily},
  keywordstyle={\bfseries\color{keyword}},
  commentstyle=\color{comment},
  stringstyle=\color{string},
  emphstyle={\color{type}},
  emphstyle={[2]\color{emph}},
  breakatwhitespace=true,
  tabsize=3,
}
\lstdefinestyle{repl}{
  language=Haskell,
  frame=single,
  backgroundcolor={\color{background}},
}

%%%_* Metadata
%%%_* Document
\begin{document}

\title{\textbf{Programming with\\Algebraic Data Types}}
\author{Christopher League\\\subtitleFont{LIU Brooklyn}}
\date{\subtitleFont{IEEE Day\\2 October 2013}}
\maketitle

\begin{frame}
  \frametitle{Underlying theory}
  Programming languages based on the 
  \textbf{polymorphic typed $\lambda$ calculus}
  \begin{itemize}
  \item Alonzo Church (1936)
  \item Haskell Curry (1958)
  \item William Alvin Howard (1969)
  \item Jean-Yves Girard (1972)
  \item John C. Reynolds (1974)
  \item Henk Barendregt (1984)
  \end{itemize}
\end{frame}

\begin{frame}
  \frametitle{Curry-Howard isomorphism}
  A direct correspondence between \textbf{computer programs} and
  \textbf{mathematical proofs.}
  \begin{itemize}
  \item \textbf{Types} in the programming language correspond to
    \textbf{propositions} in higher-order logic.
  \end{itemize}
  \begin{center}
      \begin{tabular}{lll}
        &\bf logic&\bf type system\\
        $\wedge$ & conjunction & tuple\\
        $\vee$& disjunction & alternative\\
        $\rightarrow$& implication & function\\
        $\forall$&quantification& generic\\
      \end{tabular}
    \end{center}
\end{frame}

\begin{frame}
  \frametitle{Haskell programming language}
  \begin{itemize}
  \item First defined in 1990, as an open standard based on the
    proprietary language \textbf{Miranda} (1985)
  \item Standardized in 1998 and again in 2010
  \item Premiere implementation today is the\\\textbf{Glasgow Haskell
      Compiler} (GHC)
    \vskip1cm
  \item Live examples
  \end{itemize}
\end{frame}

\begin{frame}[fragile,fragile]
  \frametitle{Defining a binary tree}
\begin{lstlisting}
data BinaryTree a
     = Empty
     | Node (BinaryTree a) a (BinaryTree a)
     deriving Show
\end{lstlisting}
\begin{lstlisting}[style=repl]
> let t0 = Node (Node (Node Empty 2 Empty) 4
                      (Node Empty 6 Empty)) 8 Empty
\end{lstlisting}
  \includegraphics[scale=0.3]{t0.png}
\end{frame}

\begin{frame}[fragile,fragile]
  \frametitle{Binary tree operations -- size}
\begin{lstlisting}
data BinaryTree a = Empty
     | Node (BinaryTree a) a (BinaryTree a)

size :: BinaryTree a -> Integer
size Empty = 0
size (Node left value right) =
     1 + size left + size right
\end{lstlisting}
\begin{lstlisting}[style=repl]
> size t0
4
\end{lstlisting}
    \includegraphics[scale=0.25]{t0.png}
\end{frame}

\begin{frame}[fragile,fragile]
  \frametitle{Binary tree operations -- height}
\begin{lstlisting}
data BinaryTree a = Empty
     | Node (BinaryTree a) a (BinaryTree a)

height :: BinaryTree a -> Integer
height Empty = 0
height (Node left value right) =
       1 + max (height left) (height right)
\end{lstlisting}
\begin{lstlisting}[style=repl]
> height t0
3
\end{lstlisting}
    \includegraphics[scale=0.25]{t0.png}
\end{frame}

\begin{frame}[fragile,fragile]
  \frametitle{Binary tree operations -- search}
\begin{lstlisting}
search :: Ord a => BinaryTree a -> a -> Bool
search Empty _ = False
search (Node _ x _) y | y == x = True
search (Node l x _) y | y < x = search l y
search (Node _ x r) y = search r y
\end{lstlisting}
\begin{lstlisting}[style=repl]
> search t0 6
True
> search t0 5
False
\end{lstlisting}
    \includegraphics[scale=0.25]{t0.png}
\end{frame}

\begin{frame}[fragile,fragile]
  \frametitle{Binary tree operations -- insert}
\begin{lstlisting}
insert :: Ord a => BinaryTree a -> a -> BinaryTree a
insert Empty y = (Node Empty y Empty)
insert (Node l x r) y | y < x = Node (insert l y) x r
insert (Node l x r) y = Node l x (insert r y)
\end{lstlisting}
\begin{lstlisting}[style=repl]
> insert t0 5
\end{lstlisting}
    \includegraphics[scale=0.25]{t5.png}
\end{frame}

\begin{frame}[fragile]
  \frametitle{Fold over a list}
\begin{lstlisting}[style=repl]
> foldl insert Empty [2,4..14]
\end{lstlisting}
\includegraphics[scale=0.25]{t1.png}  
\end{frame}

\begin{frame}[fragile,fragile]
  \frametitle{Balanced binary tree (Red-Black)}
\begin{lstlisting}
data Color = Red | Black

data RedBlackTree a
     = Empty
     | Node Color (RedBlackTree a) a (RedBlackTree a)
\end{lstlisting}
  \begin{itemize}
  \item A node is either red or black.
\item The root is black, all leaves are black.
\item Every red node must have two black child nodes.
\item Every path from a given node to any of its descendant leaves
  contains the same number of black nodes.
  \end{itemize}
\end{frame}

\newcommand{\showanim}[2]{
  \begin{frame}
    \frametitle{Insert $#1$}
    \begin{center}
      \includegraphics[scale=0.4]{#2.png}
    \end{center}
  \end{frame}}

\showanim{4}{a001}
\showanim{3}{a002}
\showanim{2}{a003}
\showanim{1}{a004}
\showanim{0}{a005}
\showanim{9}{a006}
\showanim{8}{a007}
\showanim{7}{a008}
\showanim{6}{a009}
\showanim{5}{a010}

\begin{frame}
  \frametitle{(etc.)}
  \begin{center}
    \includegraphics[scale=0.25]{r4.png}
  \end{center}
\end{frame}

\begin{frame}[fragile,fragile]
  \frametitle{Red-Black insertion}
\begin{lstlisting}
insert :: Ord a => RedBlackTree a -> a ->
    RedBlackTree a
insert t x = Node Black a z b
  where Node _ a z b = ins t
        ins Empty = Node Red Empty x Empty
        ins (Node Black a y b) =
            if x < y then balance (ins a) y b
            else balance a y (ins b)
        ins (Node Red a y b) =
            if x < y then Node Red (ins a) y b
            else Node Red a y (ins b)
\end{lstlisting}
\end{frame}

\begin{frame}[fragile]
  \frametitle{Red-Black rebalancing}
\begin{lstlisting}
balance :: RedBlackTree a -> a -> RedBlackTree a ->
    RedBlackTree a
\end{lstlisting}
\end{frame}

\begin{frame}[fragile]
  \frametitle{Lift red nodes}
\begin{lstlisting}
balance (Node Red a x b) y (Node Red c z d) =
    Node Red (Node Black a x b) y (Node Black c z d)
\end{lstlisting}
  \begin{tabular}{cc}
  \includegraphics[scale=0.33]{z1a.png}
  &
  \includegraphics[scale=0.33]{z1b.png}
  \end{tabular}
\end{frame}

\begin{frame}[fragile]
  \frametitle{Rebalance L-L}
\begin{lstlisting}
balance (Node Red (Node Red a x b) y c) z d =
    Node Red (Node Black a x b) y (Node Black c z d)
\end{lstlisting}
  \begin{tabular}{cc}
  \includegraphics[scale=0.33]{z2a.png}
  &
  \includegraphics[scale=0.33]{z2b.png}
  \end{tabular}
\end{frame}

\begin{frame}[fragile]
  \frametitle{Rebalance L-R}
\begin{lstlisting}
balance (Node Red a x (Node Red b y c)) z d =
    Node Red (Node Black a x b) y (Node Black c z d)
\end{lstlisting}
  \begin{tabular}{cc}
  \includegraphics[scale=0.33]{z3a.png}
  &
  \includegraphics[scale=0.33]{z2b.png}
  \end{tabular}
\end{frame}

\begin{frame}
  \frametitle{Resources}
  \begin{itemize}
  \item \textbf{Source available} at
    \url{https://github.com/league/algebraic-data-talk}
  \item \textbf{Purely Functional Data Structures} by Chris Okasaki
    (Cambridge University Press, 1999)
    \url{http://amzn.com/0521663504}
  \item The Haskell Programming Language \url{http://www.haskell.org/}
  \item \LaTeX{} Document Preparation System
    \url{http://www.latex-project.org/}
  \item GraphViz -- Graph Visualization Software
    \url{http://www.graphviz.org/}
  \end{itemize}
\end{frame}

%%%%%%%%%%%%%%%%%%%%%%%%%%%%%%%%%%%%%%%%%%%%%%%%%%%%%%%%%%%%
\end{document}
%% Local variables:
%% LaTeX-command: "xelatex"
%% End:
