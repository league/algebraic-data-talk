%%%_* Document preamble
\documentclass[14pt,t,usepdftitle=false,
xcolornames=x11names,svgnames,dvipsnames]{beamer}
\usepackage{euler}
\usepackage{fontspec}
\usepackage{listings}
\usepackage{alltt}
\usepackage{ulem}
\usepackage{multirow}
%%%_* Simplest theme ever; white background, no widgets
\usetheme{default}
\setbeamertemplate{navigation symbols}{}
\setbeameroption{show notes}
\setbeamertemplate{frametitle}[default][center]
\setbeamersize{text margin left=10mm}
%%%_** Fancy fonts
\setbeamerfont{frametitle}{series=\bfseries}
\setmainfont[Mapping=tex-text]{Warnock Pro}
\setsansfont[Mapping=tex-text,Numbers={OldStyle}]{Cronos Pro}
\setmonofont[Mapping=tex-text]{Inconsolata}
\newcommand{\wackyFont}[1]{
  {\LARGE\fontspec[Mapping=tex-text]{Immi Five O Five Std} #1}}
\newcommand{\subtitleFont}[1]{{\footnotesize #1}}
\newcommand{\slideheading}[1]{
  \begin{center}
    \usebeamerfont{frametitle}
    \usebeamercolor[fg]{frametitle}#1
  \end{center}\vskip-5mm}
\usefonttheme{professionalfonts}
%%%_** Color definitions
\colorlet{comment}{Olive}
\colorlet{string}{SaddleBrown}
\colorlet{keyword}{Navy}
\colorlet{type}{Green}
\colorlet{emph}{Maroon}
\colorlet{input}{Indigo}
\colorlet{error}{DarkRed}
\colorlet{result}{LightSlateGrey}
\colorlet{background}{LightGoldenrodYellow}
\colorlet{hole}{LimeGreen}
\newcommand{\cd}[1]{\textcolor{emph}{\texttt{#1}}}
%%%_** Listings settings
% "define" Scala
\lstset{
  language=Haskell,
  columns=flexible,
  showstringspaces=false,
  basicstyle={\small\ttfamily},
  keywordstyle={\bfseries\color{keyword}},
  commentstyle=\color{comment},
  stringstyle=\color{string},
  emphstyle={\color{type}},
  emphstyle={[2]\color{emph}},
  breakatwhitespace=true,
  tabsize=3,
}

%%%_* Metadata
%%%_* Document
\begin{document}

\title{\wackyFont{Programming with Algebraic Data Types}}
\author{Christopher League\\\subtitleFont{LIU Brooklyn}}
\date{\subtitleFont{IEEE Day\\2 October 2013}}
\maketitle

\begin{frame}
  \frametitle{Theory}
  polymorphically-typed $\lambda$-calculus
\end{frame}

\begin{frame}
  \frametitle{Haskell}
\end{frame}

\begin{frame}[fragile]
  \frametitle{Blah}
\begin{lstlisting}
data Tree a = Leaf a | Branch (Tree a) (Tree a)
     deriving Show

height :: Tree a -> Integer
height (Leaf _) = 0
height (Branch l r) = 1 + max (height l) (height r)
\end{lstlisting}
\end{frame}

\begin{frame}[fragile]
  \frametitle{Lift red nodes}
\begin{lstlisting}
balance (Node Red a x b) y (Node Red c z d) =
    Node Red (Node Black a x b) y (Node Black c z d)
\end{lstlisting}
  \begin{tabular}{cc}
  \includegraphics[scale=0.33]{z1a.png}
  &
  \includegraphics[scale=0.33]{z1b.png}
  \end{tabular}
\end{frame}

\begin{frame}[fragile]
  \frametitle{Rebalance L-L}
\begin{lstlisting}
balance (Node Red (Node Red a x b) y c) z d =
    Node Red (Node Black a x b) y (Node Black c z d)
\end{lstlisting}
  \begin{tabular}{cc}
  \includegraphics[scale=0.33]{z2a.png}
  &
  \includegraphics[scale=0.33]{z2b.png}
  \end{tabular}
\end{frame}

\begin{frame}[fragile]
  \frametitle{Rebalance L-R}
\begin{lstlisting}
balance (Node Red a x (Node Red b y c)) z d =
    Node Red (Node Black a x b) y (Node Black c z d)
\end{lstlisting}
  \begin{tabular}{cc}
  \includegraphics[scale=0.33]{z3a.png}
  &
  \includegraphics[scale=0.33]{z2b.png}
  \end{tabular}
\end{frame}

\newcommand{\showanim}[2]{
  \begin{frame}
    \frametitle{Insert $#1$}
    \begin{center}
      \includegraphics[scale=0.4]{#2.png}
    \end{center}
  \end{frame}}

\showanim{4}{a001}
\showanim{3}{a002}
\showanim{2}{a003}
\showanim{1}{a004}
\showanim{0}{a005}
\showanim{9}{a006}
\showanim{8}{a007}
\showanim{7}{a008}
\showanim{6}{a009}
\showanim{5}{a010}

\begin{frame}
  \frametitle{(etc.)}
  \begin{center}
    \includegraphics[scale=0.25]{r4.png}
  \end{center}
\end{frame}

\begin{frame}
  \frametitle{Resources}
  \begin{itemize}
  \item Source 
  \item \textbf{Purely Functional Data Structures} by Chris Okasaki
    (Cambridge University Press, 1999)
  \item The Haskell Programming Language \url{http://www.haskell.org/}
  \item \LaTeX{} Document Preparation System
    \url{http://www.latex-project.org/}
  \item GraphViz -- Graph Visualization Software
    \url{http://www.graphviz.org/}
  \end{itemize}
\end{frame}

%%%%%%%%%%%%%%%%%%%%%%%%%%%%%%%%%%%%%%%%%%%%%%%%%%%%%%%%%%%%
\end{document}
%% Local variables:
%% LaTeX-command: "xelatex"
%% End:
